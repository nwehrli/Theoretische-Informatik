\documentclass[a4paper, 11pt]{article}
\usepackage[a4paper, total={7in, 8in}]{geometry}

%packages
\usepackage{kantlipsum}
\usepackage{pgfplots}
\usepackage{fancyhdr}
\usepackage{lastpage}
\usepackage{mathtools}
\usepackage{tikz}
\usetikzlibrary{arrows,automata,positioning}
\usepackage{enumitem}
\usepackage{amsfonts}
\usepackage{scalerel}
\usepackage{amssymb}
\usepackage{wasysym}
\usepackage{amsmath}
\usepackage{listings}
\usepackage{algorithm}
\usepackage{algpseudocode}
\usepackage{turnstile}


%------------------------------------------------------
%Custom Commands
\def\Z{\mathbb{Z}}
\def\R{\mathbb{R}}
\def\N{\mathbb{N}}
\def\C{\mathbb{C}}
\def\L{\mathcal{L}}
\def\O{\mathcal{O}}
\def\Lre{\mathcal{L}_\text{RE}}
\def\Lr{\mathcal{L}_\text{R}}
%-----------------------------------------------------

\DeclareMathOperator*{\Bigcdot}{\scalerel*{\cdot}{\bigodot}}

\setlength{\parindent}{0mm}
\setlength{\parskip}{2mm}

\pgfplotsset{samples=100}
\pgfplotsset{compat=1.18}

%--------------------------------------------------
% HEAD
\pagestyle{fancy}
\setlength{\headheight}{5.5pt}
\rhead{Nicolas Wehrli}
\lhead{ETH Zürich, Autumn 2022}
% page numbering
\cfoot{Seite \thepage\  von \pageref{LastPage}}
%--------------------------------------------------

\begin{document}
    %--------------------------------------------------
    % Caption
    \begin{center}
        \LARGE {Theoretische Informatik \\ Beweisideen 101}
        \vspace{15pt}
        \hrule
    \end{center}
    %--------------------------------------------------
    %--------------------------------------------------
    %--------------------------------------------------
   
    \vspace{5pt}
    \section*{Kapitel 2}
    \subsection*{Lemma 2.5}
    Für jede Zahl $n \in \N$ existiert ein Wort $w_n \in  (\Sigma_{bool})^n$, so dass $$K(w_n) \geq |w_n| = n$$
    d.h., es existiert für jede Zahl $n$ ein nichtkomprimierbares Wort der Länge $n$.

    \textbf{Beweis: }
    
    Es gibt $2^n$ Wörter $x_1, ..., x_{2^n}$ über $\Sigma_{bool}$ der Länge $n$. Wir bezeichnen $C(x_i)$ als den Bitstring des kürzesten Programms, der $x_i$ generieren kann. Es ist klar, dass für $i \neq j: C(x_i) \neq C(x_j)$.
    
    Die Anzahl der Bitstrings, i.e. der Wörter der Länge $< n$ über $\Sigma_{bool}$ ist:
    $$\sum_{i = 1}^{n-1} 2^i = 2^n - 2 < 2^n$$
    Also muss es unter den Wörtern $x_1, ...,x_{2^n}$ mindestens ein Wort $x_k$ mit $K(x_k) \geq n$ geben.
    
    \hspace*{0pt}\hfill$\blacksquare$

    \subsection*{Satz 2.2}
    Sei $L$ eine Sprache über $\Sigma_{bool}$. Sei, für jedes $n \in \N \setminus \{0\}$, $z_n$ das $n$-te Wort in $L$ bezüglich der kanonischen Ordnung. Wenn ein Programm $A_L$ existiert, dass das Entscheidungsproblem $(\Sigma_{bool}, L)$ löst, dann gilt für alle $n \in \N \setminus \{0\}$, dass 
    $$K(z_n) \leq \lceil \log_2(n+1)\rceil + c$$
    wobei $c$ eine von $n$ unabhängige Konstante ist.

    \textbf{Beweisidee: } 

    Wir können aus $A_L$, ein Programm entwerfen, dass das kanonisch $n$-te Wort generiert, indem wir in der kanonischen Reihenfolge alle Wörter $x \in (\Sigma_{bool})^*$ durchgehen und mit $A_L$ entscheiden, ob $x \in L$. Dann können wir einen Counter $c$ haben und den Prozess abbrechen, wenn der Counter $c = n$ wird und dann dieses Wort ausgeben.

    Wir sehen, dass dieses Programm ausser der Eingabe $n$ immer gleich ist. Sei die Länge dieses Programms $c$, dann können wir für das $n$-te Wort der Sprache $L, z_n,$ die Kolmogorov-Komplexität auf $n$ reduzieren, bzw:
    $$K(z_n) \leq \lceil \log_2(n+1)\rceil + c$$
    \hspace*{0pt}\hfill$\blacksquare$

    \subsection*{Lemma 2.6}
    Sei $n_1, n_2, n_3, ...$ eine steigende unendliche Folge natürlicher Zahlen mit $K(n_i) \geq \lceil \log_2 n_i \rceil / 2$. Für jedes $i \in \N \setminus \{0\}$ sei $q_i$ die grösste Primzahl, die die Zahl $n_i$ teilt. Dann ist die Menge $Q = \{q_i \mid i \in \N \setminus\{0\}\}$ unendlich.

    \textbf{Beweis: }
    Wir beweisen diese Aussage per Widerspruch:

    Nehmen wir zum Widerspruch an, dass die Menge $Q = \{q_i \mid i \in \N \setminus \{0\}\}$ sei endlich. Sei $q_m$ die grösste Primzahl in $Q$. Dann können wir jede Zahl $n_i$ eindeutig als 
    $$n_i = q_1^{r_{i, 1}} \cdot q_2^{r_{i,2}} \cdot \cdots \cdot q_m^{r_{i,m}}$$
    für irgendwelche $r_{i,1}, r_{i,2}, ..., r_{i,m} \in \N$ darstellen. Sei $c$ die binäre Länge eines Programms, dass diese $r_{i,j}$ als Eingaben nimmt und $n_i$ erzeugt (A ist für alle $i\in \N$ bis auf die Eingaben $r_{i,1}, ..., r_{i,m}$ gleich).
    
    Dann gilt:
    $$K(n_i) \leq c + 8 \cdot (\lceil \log_2(r_{i,1}+1)\rceil + \lceil \log_2(r_{i,2}+1)\rceil + ... + \lceil \log_2(r_{i,m}+1)\rceil)$$
    Die multiplikative Konstante $8$ kommt daher, dass wir für die Zahlen $r_{i,1}, r_{i,2}, ..., r_{i,m}$ dieselbe Kodierung, wie für den Rest des Programmes verwenden (z.B. ASCII-Kodierung), damit ihre Darstellungen eindeutig voneinander getrennt werden können. Weil  $r_{i,j} \leq \log_2 n_i, \forall j \in \{1, ..., m\}$ erhalten wir 
    $$K(n_i) \leq c + 8m \cdot \lceil \log_2(\log_2 n_i + 1)\rceil, \forall i \in \N \setminus \{0\}$$
    Weil $m$ und $c$ Konstanten unabhängig von $i$ sind, kann 
    $$\lceil \log_2 n_i \rceil / 2 \leq K(n_i) \leq c + 8m \cdot \lceil \log_2(\log_2 n_i + 1)\rceil$$
    $$\lceil \log_2 n_i \rceil/2 \leq c + 8m \cdot \lceil \log_2(\log_2 n_i + 1)\rceil$$
    nur für endlich viele $i \in \N \setminus \{0\}$ gelten. 

    Dies ist ein Widerspruch! 

    Folglich ist die Menge $Q$ unendlich.

    \hspace*{0pt}\hfill$\blacksquare$


    \section*{Kapitel 3}
    \subsection*{Lemma 3.3}
    Sei $A = (Q, \Sigma, \delta_A, q_0, F)$ ein EA. Seien $x, y \in \Sigma^*, x \neq y$, so dass 
    $$\hat{\delta}_A(q_0, x) = p = \hat{\delta}_A(q_0, y)$$
    für ein $p \in Q$ (also $x,y \in \text{Kl[$p$]}$). Dann existiert für jedes $z \in \Sigma^*$ ein $r \in Q$, so dass $xz$ und $yz \in$ Kl[$r$], also gilt insbesondere 
    $$xz \in L(A) \iff yz \in L(A)$$
    
    \textbf{Beweis: }
    
    Aus der Existenz der Berechnungen 

    $(q_0, x) \sststile{A}{*} (p, \lambda)$ und $(q_0, y) \sststile{A}{*} (p, \lambda)$
    von $A$ folgt die Existenz der Berechnungen auf $xz$ und $yz$:

    $(q_0, xz) \sststile{A}{*} (p, z)$ und $(q_0, yz) \sststile{A}{*} (p, z)$ für alle $z \in \Sigma^*$.

    Wenn $r = \hat{\delta}_A(p, z)$ ist, dann ist die Berechnung von $A$ auf $xz$ und $yz$:

    $(q_0, xz) \sststile{A}{*} (p, z) \sststile{A}{*} (r, \lambda)$ und $(q_0, yz) \sststile{A}{*} (p, z) \sststile{A}{*} (r, \lambda)$.

    Wenn $r \in F$, dann sind beide Wörter $xz$ und $yz$ in $L(A)$. Falls $r \notin F$, dann sind $xz, yz \notin L(A)$.

    \hspace*{0pt}\hfill$\blacksquare$

    \subsection*{Lemma 3.4: Pumping-Lemma}
    Sei $L$ regulär. Dann existiert eine Konstante $n_0 \in \N$, so dass sich jedes Wort $w \in \Sigma^*$ mit $|w| \geq n_0$ in drei Teile $y, x$ und $z$ zerlegen lässt, das heisst $w = yxz$, wobei
    \begin{enumerate}[label = (\roman*)]
        \item $|yx| \leq n_0$,
        \item $|x| \geq 1$ und
        \item entweder $\{yx^kz \mid k \in \N\} \subseteq L$ oder $\{yx^kz \mid k \in \N\} \cap L = \emptyset$.
    \end{enumerate}
    \textbf{Beweis: }

    Sei $L \in \Sigma^*$ regulär. Dann existiert ein EA $A= (Q, \Sigma, \delta_A, q_0, F)$, so dass $L(A) = L$.
    \\Sei $n_0 = |Q|$ und $w \in \Sigma^*$ mit $|w| \geq n_0$. Dann ist $w = w_1w_2...w_{n_0}u$, wobei $w_i \in \Sigma$ für $i = 1, ..., n_0$ und $u \in \Sigma^*$. Betrachten wir die Berechnung auf $w_1w_2...w_{n_0}$:

    $$(q_0, w_1w_2w_3...w_{n_0}) \sststile{A}{} (q_1, w_2w_3...w_{n_0}) \sststile{A}{} (q_2, w_3...w_{n_0}) \sststile{A}{} ... \sststile{A}{} (q_{n_0-1}, w_{n_0}) \sststile{A}{} (q_{n_0}, \lambda)$$

    In dieser Berechnung kommen $n_0 + 1$ Zustände $q_0,q_1, ..., q_{n_0}$ vor. Da $|Q| = n_0$, existieren $i, j \in \{0, 1, ..., n_0\}, i < j$, so dass $q_i = q_j$. Daher haben wir in der Berechnung die Konfigurationen
    $$(q_0, w_1w_2w_3...w_{n_0}) \sststile{A}{*} (q_i, w_{i+1}w_{i+2}...w_{n_0}) \sststile{A}{*} (q_i, w_{j+1}...w_{n_0}) \sststile{A}{*} (q_{n_0}, \lambda)$$
    Dies impliziert
    $$(q_i, w_{i+1}w_{i+2}...w_j) \sststile{A}{*} (q_i, \lambda) \qquad (1)$$
    Wir setzen nun $y = w_1...w_i$, $x = w_{i+1}...w_j$ und $z = w_{j+1}...w_{n_0}u$, so dass $w = yxz$.

    Wir überprüfen nun die Eigenschaften (i),(ii) und (iii):
    \begin{enumerate}[label = (\roman*)]
        \item $yx = w_1...w_iw_{i+1}...w_j$ und daher $|yx| = j \leq n_0$.
        \item Da $|x| \geq j-i$ und $i < j$, ist $|x| \geq 1$.
        \item (1) impliziert $(q_i, x^k) \sststile{A}{*} (q_i, \lambda)$ für alle $k \in \N$.
        Folglich gilt für alle $k \in \N$:
        $$(q_0, yx^kz) \sststile{A}{*} (q_i, x^kz) \sststile{A}{*} (q_i, z) \sststile{A}{*} (\hat{\delta}_A(q_i, z), \lambda)$$
        Wir sehen, dass für alle $k \in \N$ die Berechnungen im gleichen Zustand $q_{end} = \hat{\delta}_A(q_i, z)$ enden. Falls also $q_{end} \in F$, akzeptiert $A$ alle Wörter aus $\{yx^kz \mid k \in \N\}$. Falls $q_{end}\notin F$, dann akzeptiert $A$ kein Wort aus $\{yx^kz \mid k \in \N\}$.
    \end{enumerate}
    \hspace*{0pt}\hfill$\blacksquare$

    \subsection*{Lemma 3.6}
    Sei $L_k = \{x1y \mid x \in (\Sigma_{bool})^*, y \in (\Sigma_{bool})^k\}$.

    Für alle $k \in \N\setminus\{0\}$ muss jeder EA, der $L_k$ akzeptiert, mindestens $2^k$ Zustände haben.

    \textbf{Beweis: }

    Sei $B_k = (Q_k, \Sigma_{bool}, \delta_k, q_{0k}, F_k)$ ein EA mit $L(B_k) = L_k$. 
    
    Nach \textbf{Lemma 3.3} gilt für $x,y \in (\Sigma_{bool})^*$:

    Wenn $\hat{\delta}_k (q_{0k}, x) = \hat{\delta}_k (q_{0k}, y)$, dann gilt für alle $z \in (\Sigma_{bool})^*$:
    $$xz \in L(B_k) \iff yz \in L(B_k)$$

    Die Idee des Beweises ist es, eine Menge $S_k$ von Wörtern zu finden, so dass für keine zwei unterschiedlichen Wörter $x, y \in S_k$ die Gleichung $\hat{\delta}_k (q_{0k}, x) = \hat{\delta}_k (q_{0k}, y)$ gelten darf. 
    Dann müsste $B_k$ mindestens $|S_k|$ viele Zustände haben.

    Wir wählen $S_k = (\Sigma_{bool})^k$ und zeigen, dass $\hat{\delta}_k(q_{0k}, q)$ paarweise unterschiedliche Zustände für alle $u \in  S_k$ sind. 

    Wir beweisen dies per Widerspruch. 

    Seien $x = x_1x_2...x_k$ und $y = y_1y_2...y_k$ für $x_i,y_i \in \Sigma_{bool}, i \in \{1, ..., k\}$ zwei unterschiedliche Wörter aus $S_k$.

    Nehmen wir zum Widerspruch an, dass $\hat{\delta}_k(q_{0k}, x) = \hat{\delta}_k(q_{0k}, y)$.
    
    Weil $x \neq y$, existiert ein $j \in \{1, ...,k\}$, so dass $x_j \neq y_j$. O.B.d.A. setzen wir $x_j = 1$ und $y_j = 0$. 
    Betrachten wir nun $z = 0^{j-1}$.  Dann ist 

    $xz = x_1...x_{j-1}1x_{j+1}...x_k0^{j-1}$ und $yz = y_1...y_{j-1}0y_{j+1}...y_k0^{j-1}$

    und daher $xz \in L_k$ und $yz \notin L_k$. Dies ist ein Widerspruch! Folglich gilt $\hat{\delta}_k(q_{0k}, x) \neq \hat{\delta}_k(q_{0k}, y)$ für alle paarweise unterschiedliche $x,y \in S_k = (\Sigma_{bool})^k$.
    
    Daher hat $B_k$ mindestens $|S_k| = 2^k$ viele Zustände.
    
    \hspace*{0pt}\hfill$\blacksquare$

    \section*{Kapitel 4}
    \subsection*{Lemma 4.2}
    Für jede Mehrband-TM $A$ existiert eine zu $A$ äquivalente TM $B$.

    \textbf{Beweis: }

    Sei $A$ eine $k$-Band-Turingmaschine für ein $k \in \N \setminus \{0\}$. Wir konstruieren eine TM $B$, die Schritt für Schritt $A$ simuliert.

    $B$ speichert die Inhalte aller $k+1$ Bänder von $A$ auf ihrem einzigen Band. Anschaulich gesprochen ist jedes Feld auf dem Band von $B$ ein $2(k+1)$-Tupel und jedes Element dieses Tupels ist auf einer Spur. 
    Sei $\Gamma_A$ das Arbeitsalphabet von $A$. Dann gilt 
    $$\Gamma_B = (\Sigma_A \cup \{\cent, \$, \textvisiblespace\}) \times \{\textvisiblespace,\uparrow\} \times (\Gamma_A \times \{\textvisiblespace, \uparrow\})^k \cup \Sigma_A \cup \{\textvisiblespace, \cent\}$$
    Für ein Symbol $\alpha = (a_0,a_1,a_2,...,a_{2k+1}) \in \Gamma_B$ sagen wir, dass $a_i$ auf der $i$-ten Spur liegt. Daher bestimmen die $i$-ten Elemente der Symbole auf dem Band von $B$ den Inhalt der $i$-ten Spur. Eine Konfiguration $(q,w,i,x_1,i_1,x_2,i_2,...,x_k,i_k)$ von $A$ ist dann in $B$ wie folgt gespeichert. 
    \begin{itemize}
        \item Der Zustand $q$ ist in der endlichen Kontrolle von $B$ gespeichert. 
        \item Die $0$-te Spur des Bandes von $B$ enthält die $\cent w\$$ (i.e. den Inhalt des Eingabebandes von $A$)
        \item Für alle $i \in \{1, ..., k\}$ enthält die $(2i)$-te Spur des Bandes von $B$ den Inhalt vom $i$-ten Band von $A$ (i.e. $\cent x_i\$$).
        \item Für alle $i \in \{1, ..., k\}$ bestimmt die $(2i +1)$-te Spur des Bandes von $B$ mit dem Symbol $\uparrow$ die Position des Kopfes auf dem $i$-ten Arbeitsband von $A$.
    \end{itemize}
    Ein Schritt von $A$ kann jetzt durch folgende Prozedur von $B$ simuliert werden:
    \begin{enumerate}
        \item $B$ liest einmal den Inhalt ihres Bandes von links nach rechts, bis sie alle $k+1$ Kopfpositionen von $A$ gefunden hat, und speichert dabei in ihrem Zustand die $k+1$ Symbole, die an diesen Positionen stehen. (Dies kann ohne weiteres in der Zustandsmenge abgespeichert werden, da $k$ fix ist, folglich ist dann $\Gamma_A^k$ auch endlich)
        \item Nach der ersten Phase kennt $B$ das ganze Argument (der Zustand von $A$ ist im Zustand von $B$ gespeichert) der Transitionsfunktion von $A$ und kann also die entsprechenden Aktionen (Köpfe bewegen, Ersetzen von Symbolen) von $A$ bestimmen. Diese Änderungen führt $B$ in einem Lauf über ihr Band von rechts nach links durch.
    \end{enumerate}
    \hspace*{0pt}\hfill$\blacksquare$


    \section*{Kapitel 5}

    \subsection*{Satz 5.4}
    $\mathcal{P}((\Sigma_{bool})^*)$ ist nicht abzählbar.

    \textbf{Beweis: }

    Wir definieren eine injektive Funktion von $f: [0, 1] \to \mathcal{P}((\Sigma_{bool})^*)$ und beweisen so $|\mathcal{P}((\Sigma_{bool})^*)| \geq |[0, 1]|$.

    Sei $a \in [0, 1]$ beliebig. Wir können $a$ wie folgt binär darstellen:
    Nummer$(a) = 0.a_1a_2a_3a_4...$ mit $a = \sum_{i = 1}^{\infty} a_i\cdot 2^{-i}$. Hier ist zu beachten, dass wir für eine Zahl $a$ immer die lexikographisch letzte Darstellung. Dies tun wir, weil eine reelle Zahl 2 verschiedene Binärdarstellungen haben kann. Beispiel: $\frac{1}{2} = 0.1\overline{0} = 0.0\overline{1}$.
    
    Für jedes $a$ definieren wir:
    $$f(a) = \{a_1, a_2a_3, a_4a_5a_6, ..., a_{\binom{n}{2}+1}a_{\binom{n}{2}+2}...a_{\binom{n+1}{2}} , ...\}$$

    Da $f(a) \subseteq (\Sigma_{bool})^*$ gilt $f(a) \in \mathcal{P}((\Sigma_{bool})^*)$.

    Wir haben für alle $n \in \N \setminus\{0\}$, dass $f(a)$ \textbf{genau} ein Wort dieser Länge enthält. Nun können wir daraus folgendes schliessen:

    Weil die Binärdarstellung zweier unterschiedlichen reellen Zahlen an mindestens einer Stelle unterschiedlich ist, gilt $b \neq c \implies f(b) \neq f(c), \forall b,c \in [0, 1]$. 

    Folglich ist $f$ injektiv und wir haben $|\mathcal{P}((\Sigma_{bool})^*)| \geq |[0, 1]|$.

    Da $[0,1]$ nicht abzählbar ist, folgt daraus:

    $\mathcal{P}((\Sigma_{bool})^*)$ ist nicht abzählbar.
    
    \hspace*{0pt}\hfill$\blacksquare$

    \subsection*{Satz 5.5}
    $L_{\text{diag}} \notin \L_{\text{RE}}$.

    \textbf{Beweis:}

    Wir haben 
    $$L_{\text{diag}} = \{w \mid w = w_i \text{ und $M_i$ akzeptiert $w_i$ nicht für ein $i \in \N\setminus\{0\}$}\}$$

    Widerspruchsbeweis:

    Sei $L_\text{diag} \in \L_{\text{RE}}$. Dann existiert eine TM $M$, so dass $L(M) = L_\text{diag}$. Da diese TM eine TM in der Nummerierung aller TM ist, existiert ein $i \in \N$, so dass $M_i = M$.

    Wir betrachten nun das Wort $w_i$ für diese $i \in \N$. Per Definition von $L_\text{diag}$, gilt:

    $$w_i \in L_\text{diag} \iff w_i \notin L(M_i)$$

    Da aber $L(M_i) = L_\text{diag}$, haben wir folgenden Widerspruch:
    $$w_i \in L_\text{diag} \iff w_i \notin L_\text{diag}$$

    Folglich gilt $L_\text{diag} \notin \L_\text{RE}$.

    \hspace*{0pt}\hfill$\blacksquare$

    \subsection*{Lemma 5.4}
    Sei $\Sigma$ ein Alphabet. Für jede Sprache $L \subseteq \Sigma^*$ gilt:
    
    $$L \leq_\text{R} L^\complement \text{ und } L^\complement \leq_\text{R} L$$

    \textbf{Beweis: }
    Es reicht $L^\complement \leq_\text{R} L$ zu zeigen, da $(L^\complement)^\complement = L$ und somit dann $(L^\complement)^\complement = L \leq_\text{R} L^\complement$.

    Sei $M'$ ein Algorithmus für $L$, der immer hält ($L \in \L_\text{R}$). Dann beschreiben wir einen Algorithmus $B$, der $L^\complement$ entscheidet. 
    
    $B$ übernimmt die Eingaben und gibt sie an $M'$ weiter und invertiert dann die Entscheidung von $M'$. Weil $M'$ immer hält, hält auch $B$ immer und wir haben offensichtlich $L(B) = L$.

    \hspace*{0pt}\hfill$\blacksquare$

    \subsection*{Korollar 5.2 (bzw. Anwendung von Lemma 5.4)}
    $(L_\text{diag})^\complement \notin \L_\text{R}$.
    
    \textbf{Beweis:} 
    
    Aus Lemma 5.4 haben wir $L_\text{diag} \leq_\text{R} (L_\text{diag})^\complement$. Daraus folgt $L_\text{diag} \notin \Lr \implies (L_\text{diag})^\complement \notin \Lr$.
    Da $L_\text{diag} \notin \Lre$ gilt auch $L_\text{diag} \notin \Lr$. 
    
    Folglich gilt $(L_\text{diag})^\complement \notin \Lr$.

    \hspace*{0pt}\hfill$\blacksquare$

    
   
    \subsection*{Lemma 5.8}
    $L_{\text{H}, \lambda} \notin \Lr$.

    \textbf{Beweis: }

    Wir zeigen $L_\text{H} \leq_\text{EE} L_{\text{H}, \lambda}$. Wir beschreiben einen Algorithmus $B$, so dass $x \in L_\text{H} \iff B(x) \in L_{\text{H}, \lambda}$.

    Für jede Eingabe arbeitet $B$ wie folgt:
    \begin{itemize}
        \item Falls $x$ von der falschen Form, dann $B(x) = M_{inf}$, wobei $M_{inf}$ unabhängig von der Eingabe immer unendlich läuft.
        \item Sonst $x = \text{Kod}(M)\#w$: Dann $B(x) = M'$, wobei $M'$ die Eingabe ignoriert und immer $M$ auf $w$ simuliert.
    \end{itemize}

    Wir sehen, dass $M'$ genau dann auf $\lambda$ hält, wenn $x \in L_{\text{H}}$.

    Daraus folgt $x \in L_\text{H} \iff B(x) \in L_{\text{H}, \lambda}$.

    \hspace*{0pt}\hfill$\blacksquare$

    \section*{Kapitel 6}
    
    \subsection*{Lemma 6.1}
    Sei $k$ eine positive ganze Zahl. Für jede $k$-Band Turingmaschine $A$, die immer hält, existiert eine äquivalente $1$-Band-TM $B$, so dass
    $$\text{Space}_B(n) \leq \text{Space}_A(n)$$

    \textbf{Beweisskizze: }

    Gleiche Konstruktion wie in Lemma 4.2. Wir können leicht sehen, dass $B$ genau so viele Felder braucht, wie $A$.


    \subsection*{Lemma 6.2}
    Zu jeder MTM $A$ existiert eine äquivalente MTM $B$ mit 
    $$\text{Space}_B(n) \leq \frac{\text{Space}_A(n)}{2}+2$$

    \textbf{Beweisskizze: }

    Wir fassen jeweils 2 Felder von $A$ zu einem Feld in $B$ zusammen. $\Gamma_B = \Gamma_A \times \Gamma_A$. Wir addieren $1$ für das $\cent$ am linken Rand und $1$ für das Aufrunden im Fall von ungerader Länge.

    \subsection*{Lemma 6.3}
    TIME($t$) $\subseteq$ SPACE($t$)

    \textbf{Beweisskizze: }
    In $t$ Schritten sind höchstens $t$ Felder beschreibbar.

    \subsection*{Lemma 6.4}
    Sei $S$ platzkonstruierbar. Für jede MTM $M$, für welche Space$_M(w) \leq s(|w|)$ nur für alle $w \in L(M)$ erfüllt, existiert eine äquivalente MTM $M'$, welche dies für alle $w \in \Sigma^*$ erfüllt.

    \textbf{Beweisskizze: }
    Erzeuge für jede Eingabe $x \in \Sigma^*$ zuerst $0^{s(|x|)}$ auf einem zusätzlichen Band und nutze das als Platzüberwachung. Wenn $M'$ diesen Platz überschreiten will, wird die Simulation unterbrochen und die Eingabe verworfen.

    \subsection*{Lemma 6.5}
    Sei $t$ zeitkonstruierbar. Zu jeder MTM, welche Time$_M(w) \leq t(|w|)$ nur für alle $w \in L(M)$ erfüllt, existiert eine äquivalente MTM $M'$, welche zumindest Time$_M(w) \leq 2t(|w|) \in \O(t(|w|))$ für alle $w \in \Sigma^*$ erfüllt.

    \textbf{Beweisskizze: }
    Schreibe für jede Eingabe $x \in \Sigma^*$ $0^{t(|x|)}$ auf ein zusätzliches Arbeitsband und nutze dies zur Zeitzählung. Wenn $M'$ mehr Schritte machen will, wird die Simulation abgebrochen und die Eingabe verworfen.

    \subsection*{Satz 6.2}
    Für jede Funktion $s$ mit $s(n) \geq \log_2(n)$ gilt:

    $$\textbf{SPACE}(s(n)) \subseteq \bigcup_{c \in \N} \textbf{TIME}(c^{s(n)})$$

    \textbf{Beweis: } 

    Sei $L \in \textbf{SPACE}(s(n))$. Nach Lemma 6.1 existiert eine 1-Band-TM $M=(Q,\Sigma,\Gamma,\delta,q_0,q_{accept},q_{reject})$, die \textbf{immer hält}, so dass $L = L(M)$ und Space$_M(n) \leq d \cdot s(n)$ für $d \in \N$ gelten. Für jede Konfiguration $C =(q,w,i,x,j)$ von $M$ definieren wir die \textbf{innere Konfiguration von $C$} als $$\text{In}(C) = (q,i,x,j).$$ 
    Die innere Konfiguration enthält das Eingabewort $w$ nicht, da dies sich während einer Berechnung nicht ändert. 

    Wir betrachten die Menge aller inneren Konfigurationen , dass bei einer \textbf{deterministischen} TM jede Berechnung $D = C_1,C_2,C_3, ...$ von $M$ auf einem Wort $w$ mit $|w| = n$, die länger als 





%-----------------------------------------------------------------



%------------------------------------------------------------------

    \section{EE-Reduktionen und R-Reduktionen -- Komplexitätsbeweise}
    \textit{Mit Inspiration von der Zsf. von Fabian Frei}

    Generelle Bemerkungen:
    \begin{itemize}
        \item $L$ rekursiv (entscheidbar) $\iff$ $L \in \L_{\text{R}}$
        \item $L$ rekursiv aufzählbar $\iff$ $L \in \L_{\text{RE}}$
        \item "Algorithmus" ist ein anderes Wort für eine Turingmaschine, die \textbf{immer} terminiert.
    \end{itemize}
    \subsection{$L \in \L_\text{R}$ }
    Wir kennen zwei Methoden um dies zu beweisen:
    \begin{itemize}
        \item Wir finden eine Sprache $L' \in \L_\text{R}$ und zeigen $L \leq_{\text{R}} L'$. (Meistens ein wenig umständlich)
        \item Direkter Beweis:  Eine TM (bzw. ein Algorithmus) $A$ beschreiben, so dass $L(A) = L$ und $A$ immer terminiert.
    \end{itemize}
    \subsection{$L \notin \L_{\text{R}}$}
    Wir kennen hier auch 3 Arten:
    \begin{itemize}
        \item Folgt sofort aus $L \notin \L_{\text{RE}}$, da $\L_{\text{R}} \subset \L_{\text{RE}}$.
        \item Wir wählen eine Sprache $L'$, so dass $L' \notin \L_{\text{R}}$ und beweisen $L' \leq_{\text{R/EE}} L$.
        
            Geeignete Sprachen als $L'$ sind: $L_{empty}^\complement, L_{diag}^\complement, L_\text{H}, L_\text{U}, L_{\text{H}, \lambda}$. (Alle im Buch bewiesen)
        \item Satz von Rice 
    \end{itemize}

    Für den \textbf{Satz von Rice}:
    \begin{itemize}
        \item Wir können mit diesem Satz nur $L \notin \L_{\text{R}}$ beweisen!
        \item Wir haben folgende Bedingungen:
        \begin{enumerate}
            \item $L \subseteq \text{KodTM}$
            \item $\exists$ TM $M$: Kod$(M) \in L$
            \item $\exists$ TM $M$: Kod$(M) \notin L$
            \item $\forall$ TM $M_1, M_2$: $L(M_1) = L(M_2) \implies \left(\text{Kod}(M_1) \in L \iff \text{Kod}(M_2) \in L\right)$
        \end{enumerate}
        Für den letzten Punkt (4) muss man überprüfen, ob in der Definition von $L = \{\text{Kod}(M) \mid M \text{ ist TM und ...}\}$ überall nur $L(M)$ vorkommt und nirgends $M$ direkt. Beziehungsweise reicht es, wenn man die Bedingung so umschreiben kann, dass sie nur noch durch $L(M)$ beschrieben ist.
    \end{itemize}
    
    
    \subsection{$L \in \L_{\text{RE}}$}
    Wir beschreiben eine TM $M$ mit $L(M) = L$, die nicht immer halten muss. 
    
    Meistens muss die TM eine Eigenschaft, für alle möglichen Wörter prüfen. (Bsp: $\text{Kod}(M_1) \in L_\text{H}^\complement$: Wir gehen alle Wörter durch, um dasjenige zu finden, für das $M_1$ hält.)
    
    Wir verwenden oft einen von den folgenden 2 Tricks, um dies zu tun:
    \begin{itemize}
        \item Da es für jede NTM $M'$, eine TM $M$ gibt, so dass $L(M') = L(M)$, können wir eine solche definieren, für die $L(M') = L$ gilt.
        \item Die andere Variante, ist die parallele Simulation von Wörtern, bei dem man das Diagonalisierungsverfahren aus dem Buch verwendet. (Bsp: Beweis $L_{\text{empty}} \in \L_{\text{RE}}$, S. 156 Buch)
    \end{itemize}

    \subsection{$L \notin \L_{\text{RE}}$}
    Hier haben wir 2 mögliche (offizielle) Methoden:
    \begin{itemize}
        \item Diagonalisierungsargument mit Widerspruch, wie beim Beweis von $L_{\text{diag}} \notin \L_{\text{RE}}$.
        \item Widerspruchsbeweis mit der Aussage $L \in \L_{\text{RE}} \land L^\complement \in \L_{\text{RE}} \implies L \in \L_{\text{R}}$.
    \end{itemize}
    Inoffiziell könnten wir auch die EE-Reduktion verwenden, wird aber weder in der Vorlesung noch im Buch erwähnt.

    \subsection{EE- und R-Reduktionen: Tipps und Tricks}
    \begin{itemize}
        \item Die vorgeschaltete TM $A$ muss immer terminieren! I.e. sie muss ein Algorithmus sein.
        \item Die Eingabe sollte immer zuerst auf die Richtige Form überprüft werden!
        
        Auch im Korrektsheitsbeweis, sollte dieser Fall als erstes abgehandelt werden.
        \item Für Korrektheit müssen wir immer $x \in L_1 \iff A(x) \in L_2$ beweisen.
        \item Wir verwenden meistens folgende 2 Tricks:
        \begin{enumerate}
            \item Transitionen nach $q_{accept}$ oder $ q_{reject}$ umleiten nach $q_{reject}$/$q_{accept}$ oder einer \textbf{Endlosschleife}. 
            \item TM $M'$ konstruieren, die ihre Eingabe ignoriert und immer dasselbe tut (z.B. eine TM dessen Kodierung gegeben ist, auf ein fixes Wort simuliern).
        \end{enumerate}
        \item Die Kodierung einer TM generieren, dessen Sprache gewisse Eigenschaften hat(z.B. sie akzeptiert alle Eingaben, läuft immer unendlich etc.)
    \end{itemize}

    



    \section{Polynomialzeitreduktionen}

    Typische Aufgabe: $L$ ist NP-Vollständig. Dann müssen wir (i) $L$ in NP und (ii) $L$ ist NP-schwer zeigen.

    \begin{enumerate}[label = (\roman*)]
        \item Wir beschreiben eine NTM $M$, so dass $L(M) = L$. $M$ errät (nichtdeterministisch) ein Zertifikat und verfiziert dies (deterministisch) in Polynomialzeit.
        $M$ akzeptiert, wenn die Verfikation erfolgreich ist.

        $M$ akzeptiert $\iff$ $M$ hat eine akzeptierende Berechnung
        \item \begin{itemize}
             \item Wir nehmen eine Sprache $L'$ die NP-Schwer ist und zeigen $L' \leq_{p} L$.
             
                \textbf{Beweisidee: }
                
                Wir zeigen eine Reduktion indem wir einen Polynomialzeit Algorithmus $A$ beschreiben, so dass $x \in L \iff A(x) \in L'$.
                Wir müssen also folgende 2 Punkte für $A$ beweisen:
                \begin{itemize}
                    \item $x \in L \iff A(x) \in L'$ (meist recht komplex, beide Richtungen einzeln beweisen)
                    \item $A$ läuft in Polynomialzeit (meist trivial, es reicht eine High-Level Begründung zu geben)
                \end{itemize}


             \item Wir könnten es auch direkt beweisen(wie Beweis vom Satz von Cook). Dies ist aber meist zu komplex.
        \end{itemize}
    \end{enumerate}

    




    \section{Grammatiken}



    \textbf{Beispiel 10.6}

    Sei $L = \{a^nb^nc^n \mid n \in \N\}$
   
    Beweis durch Widerspruch:

    Sei $L$ kontextfrei. Dann gilt das Pumping Lemma für kontextfreie Sprachen.

    Sei $n_L$  die Konstante aus dem Pumping Lemma.

    Dann wählen wir $z = a^{n_L}b^{n_L}c^{n_L}, |z| \geq n_L, z \in L$.

    Dann gilt für jede Partition $z = uvwxy$ mit (i) $|vx| \geq 1$ und (ii) $|vwx| \leq n_L$, auch (iii) $\{uv^iwx^iy \mid i \in \N\}$.
\end{document}
