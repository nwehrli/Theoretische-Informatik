            
            
            
            \section{Reduktion}
            
            \begin{itemize}
                \item Reduktionen sind klassische Aufgaben an dem Endterm. Ein bisschen wie Nichtregularitätsbeweise.
                \item Ist aber auch nicht so schlimm.
            \end{itemize}
        
        
        
            \myTitle{R-Reduktion}
            \begin{mainbox}{Definition 5.3}
                Seien $L_1 \subseteq \Sigma_{1}^*$ und $L_2 \subseteq \Sigma_{2}^*$ zwei Sprachen. Wir sagen, dass $\mathbf{L_1}$ \textbf{auf $\mathbf{L_2}$ rekursiv reduzierbar ist, $\mathbf{L_1 \leq_{\textbf{R}} L_2}$}, falls
                $$L_2 \in \Lr \implies L_1 \in \Lr$$
            \end{mainbox}
            \textbf{Bemerkung:}
        
            Intuitiv bedeutet das \textit{''$L_2$ mindestens so schwer wie $L_1$''} (bzgl. algorithmischen Lösbarkeit).
        
        
        
            \myTitle{EE-Reduktion}
            \begin{mainbox}{Definition 5.4}
                Seien $L_1 \subseteq \Sigma_{1}^*$ und $L_2 \subseteq \Sigma_{2}^*$ zwei Sprachen. 
                Wir sagen, dass $\mathbf{L_1}$ \textbf{auf $\mathbf{L_2}$ EE-reduzierbar ist, $\mathbf{L_1 \leq_{\textbf{EE}} L_2}$}, wenn eine TM $M$ existiert, 
                die eine Abbildung $f_M: \Sigma_1^* \to \Sigma_2^*$ mit der Eigenschaft
                $$x \in L_1 \iff f_M(x) \in L_2$$
                für alle $x \in \Sigma_1^*$ berechnet. Wir sagen auch, dass die TM $M$ die Sprache $L_1$ auf die Sprache $L_2$ reduziert.
            \end{mainbox}
            
            \begin{subbox}{}
                Wir sagen, dass $M$ eine Funktion $F: \Sigma^* \to \Gamma^*$ \textbf{berechnet}, 
                falls für alle $x \in \Sigma^*$: $q_0\cent x \sststile{M}{*} q_{\text{accept}}\cent F(x)$.
            \end{subbox}
            \begin{figure}[htp]
                \includegraphics[width=0.8\textwidth]{Images/Basic_EE-Reduktion.png}
                \caption{Abbildung 5.7 vom Buch}
            \end{figure}
        
        
        
            \myTitle{Verhältnis von EE-Reduktion und R-Reduktion}
            \begin{mainbox}{Lemma 5.3}
                Seien $L_1 \subseteq \Sigma_{1}^*$ und $L_2 \subseteq \Sigma_{2}^*$ zwei Sprachen.
                $$L_1 \leq_{\text{EE}} L_2 \implies L_1 \leq_{\text{R}} L_2$$
            \end{mainbox}
            \textbf{Beweis: }
            $$L_1 \leq_{\text{EE}} L_2 \implies \exists \text{TM } M. \ x \in L_1 \iff M(x) \in L_2$$
        
            Wir zeigen nun $L_1 \leq_{\text{R}} L_2$, i.e. $L_2 \in \Lr \implies L_1 \in \Lr$.
        
            Sei $L_2 \in \Lr$. Dann existiert ein Algorithmus $A$ (TM, die immer hält), der $L_2$ entscheidet.

            Wir konstruieren eine TM $B$ (die immer hält) mit $L(B) = L_1$
        
            Für eine Eingabe $x \in \Sigma_1^*$ arbeitet $B$ wie folgt:
            \begin{enumerate}[label=(\roman*)]
                \item $B$ simuliert die Arbeit von $M$ auf $x$, bis auf dem Band das Wort $M(x)$ steht.
                \item $B$ simuliert die Arbeit von $A$ auf $M(x)$.
                
                Wenn $A$ das Wort $M(x)$ akzeptiert, dann akzeptiert $B$ das Wort $x$.
        
                Wenn $A$ das Wort $M(x)$ verwirft, dann verwirft $B$ das Wort $x$.
            \end{enumerate}
            $A$ hält immer $\implies$ $B$ hält immer und somit gilt $L_1 \in \Lr$  
        
            $\hfill\blacksquare$
        
        
        
            \myTitle{$L$ und $L^\complement$}

            \begin{mainbox}{Lemma 5.4}
                Sei $\Sigma$ ein Alphabet. Für jede Sprache $L \subseteq \Sigma^*$ gilt:
             $$L \leq_\text{R} L^\complement \text{ und } L^\complement \leq_\text{R} L$$
            \end{mainbox}
            \textbf{Beweis: }
        
            Es reicht $L^\complement \leq_\text{R} L$ zu zeigen, da $(L^\complement)^\complement = L$ und somit dann $(L^\complement)^\complement = L \leq_\text{R} L^\complement$.
        
            Sei $M'$ ein Algorithmus für $L$, der immer hält ($L \in \L_\text{R}$). Dann beschreiben wir einen Algorithmus $B$, der $L^\complement$ entscheidet. 
            
            $B$ übernimmt die Eingaben und gibt sie an $M'$ weiter und invertiert dann die Entscheidung von $M'$. Weil $M'$ immer hält, hält auch $B$ immer und wir haben offensichtlich $L(B) = L$.
        
            \hspace*{0pt}\hfill$\blacksquare$
        
            
            \begin{subbox}{Korollar 5.2}
                $$(L_\text{diag})^\complement \notin \L_\text{R}$$
            \end{subbox}
            
            \textbf{Beweis:} 
            
            Aus Lemma 5.4 haben wir $L_\text{diag} \leq_\text{R} (L_\text{diag})^\complement$. Daraus folgt $L_\text{diag} \notin \Lr \implies (L_\text{diag})^\complement \notin \Lr$.
            Da $L_\text{diag} \notin \Lre$ gilt auch $L_\text{diag} \notin \Lr$. 
            
            Folglich gilt $(L_\text{diag})^\complement \notin \Lr$.
        
            \hspace*{0pt}\hfill$\blacksquare$
        
        
            
            \subsection{Beispielsprachen}
          

            \begin{mainbox}{Lemma 5.5}
                $$L_{\text{diag}}^{\complement} \in \Lre$$
            \end{mainbox}
            \textbf{Beweis}
                
            
                Direkter Beweis: Wir beschreiben eine TM $A$ mit $L(A) = L_{\text{diag}}^\complement$.
                
                \textbf{Eingabe}: $x \in (\Sigma_{\text{bool}})^*$
                
                \begin{enumerate}[label=(\roman*)]
                    \item Berechne $i$, so dass $w_i = x$ in kanonischer Ordnung
                    \item Generiere $\text{Kod}(M_i)$.
                    \item Simuliere die Berechnung von $M_i$ auf $w_i = x$. 
                \end{enumerate}
               
            
                \begin{itemize}[label=-]
                    \item Falls $w_i \in L(M_i)$ akzeptiert, akzeptiert $A$ die Eingabe $x$.
                    \item Falls $M_i$ verwirft (hält in $q_{\text{reject}}$), dann hält $A$ und verwirft $x = w_i$ auch.
                    \item Falls $M_i$ unendlich lange arbeitet, wird $A$ auch nicht halten und dann folgt auch $x \notin L(A)$.  
                \end{itemize}
                Aus dem folgt $L(A) = L_{\text{diag}}^\complement$.
            
                $\hfill\blacksquare$
                \begin{subbox}{Korollar 5.3}
                    $L_{\text{diag}}^\complement \in \Lre \setminus \Lr$ und daher $\Lr \subsetneq \Lre$.
                \end{subbox}
            
            
                \myTitle{Universelle Sprache}
                Sei 
                $$L_{\text{U}} := \{\text{Kod}(M)\#w \mid w \in (\Sigma_{\text{bool}})^* \text{ und } M \text{ akzeptiert }w\}$$
                \begin{mainbox}{Satz 5.6}
                    Es gibt eine TM $U$, \textbf{universelle TM} genannt, so dass 
                    $$L(U) = L_{\text{U}}$$
                    Daher gilt $L_{\text{U}} \in \Lre$.
                \end{mainbox} 
                \textbf{Beweis}
            
                Direkter Beweis: Konstruktion einer TM.
            
                \textit{Siehe Buch/Vorlesung.}
            
                \begin{mainbox}{Satz 5.7}
                    $$L_{\text{U}} \notin \Lr$$
                \end{mainbox}
                \textbf{Beweis}
            
                Wir zeigen $L_{\text{diag}}^\complement \leq_{\text{R}} L_{\text{U}}$.
            
                \textit{Siehe Buch/Vorlesung.}
                \myTitle{Halteproblem}
                Sei $$L_{\text{H}} = \{\text{Kod}(M)\#w \mid M \text{ hält auf }w\}$$
                \begin{mainbox}{Satz 5.8}
                    $$L_{\text{H}} \notin \Lr$$
                \end{mainbox}
                \textbf{Beweis}
            
                Wir zeigen $L_{\text{U}} \leq_{\text{R}} L_{\text{H}}$.
            
                \textit{Siehe Buch/Vorlesung.}
            
                
                \begin{mainbox}{Lemma 5.7}
                    $$L_{\text{empty}}^\complement \notin \Lr$$
                \end{mainbox}
                \textbf{Beweis }
            
                Wir zeigen $L_{\text{U}} \leq_{\text{EE}} L_{\text{empty}}^\complement$.
            
                \textit{Siehe Buch/Vorlesung.}
               
            
                \begin{subbox}{Korollar 5.4}
                    $$L_{\text{empty}} \notin \Lr$$
                \end{subbox}
                \begin{subbox}{Korollar 5.5}
                    $$L_{\text{EQ}} \notin \Lr$$
                    für $L_{\text{EQ}} = \{\text{Kod}(M)\#\text{Kod}(\overline{M}) \mid L(M) = L(\overline{M})\}$.
                \end{subbox}
            
            
            
                \subsection{Parallele Simulation vs Nichtdeterminismus}
                Sei 
                $$L_{\text{empty}} = \{\text{Kod}(M) \mid L(M) = \emptyset\}$$
                und 
                $$L_{\text{empty}}^\complement = \{\text{Kod}(M) \mid L(M) \neq \emptyset\} \cup \{x \in \{0,1, \#\}\mid x \notin \textbf{KodTM}\}$$
                \begin{mainbox}{Lemma 5.6}
                    $$L_{\text{empty}}^\complement \in \Lre$$
                \end{mainbox}

                \textbf{Nichtdeterminismus Beweis}
            
                Da für jede NTM $M_1$ eine TM $M_2$ existiert mit $L(M_1) = L(M_2)$, reicht es eine NTM $M_1$ mit $L(M_1) = L_{\text{empty}}^\complement$ zu finden.
            
                Eingabe: $x \in \{0,1,\#\}$
                \begin{enumerate}[label=(\roman*)]
                    \item $M_1$ prüft deterministisch, ob $x = \text{Kod}(M)$ für eine TM $M$. 
                    
                    Falls $x$ keine TM kodiert, wird $x$ akzeptiert.
                    
                    \item Sonst gilt $x = \text{Kod}(M)$ für eine TM $M$ und $M_1$ wählt nichtdeterministisch ein Wort $y \in (\Sigma_{\text{bool}})^*$. 
                    
                    \item Dann simuliert $M_1$ die TM $M$ auf $y$ deterministisch und übernimmt die Ausgabe. 
                \end{enumerate} 
            
            
                Wir unterscheiden zwischen 3 Fällen
                \begin{enumerate}[label = \Roman*]
                    
                    \item $x = \text{Kod}(M)$ und $L(M) = \emptyset$
                    
                    Dann gilt $x \notin L_{\text{empty}}^\complement$ und da es keine akzeptierende Berechnung gibt, auch $x \notin L(M_1)$.
                    
                    \item $x = \text{Kod}(M)$ und $L(M) \neq \emptyset$
                    
                    Dann gilt $x \in L_{\text{empty}}^\complement$ und da es eine akzeptierende Berechnung gibt, auch $x \in L(M_1)$.
                    
                    \item $x$ kodiert keine TM
                    
                    Wir haben $x \in L_{\text{empty}}^\complement$ und wegen Schritt (i) auch $x \in L(M_1)$.
                \end{enumerate}
                Somit gilt $L(M_1) = L_{\text{empty}}^\complement$.
            
                $\hfill\blacksquare$
        
                \textbf{Beweis per Parallele Simulation}
            
                Wir konstruieren eine TM $A$ mit $L(A) = L$ direkt.
                Eingabe: $x \in \{0,1,\#\}$
                \begin{enumerate}[label=\Roman*.]
                    
                    \item Falls $x$ keine Kodierung einer TM ist, akzeptiert $A$ die Eingabe.
                    
                    \item Falls $x = \text{Kod}(M)$ für eine TM $M$, arbeitet $A$ wie folgt
                    \begin{itemize}[label=-]
                        \item Generiert systematisch alle Paare $(i, j) \in (\N \setminus \{0\}) \times (\N \setminus \{0\})$. (Abzählbarkeit)
                        \item Für jedes Paar $(i, j)$, generiert $A$ das kanonisch $i$-te Wort $w_i$ und simuliert $j$ Berechnungsschritte der TM $M$ auf $w_i$.
                        \item Falls $M$ an ein Wort akzeptiert, akzeptiert $A$ das Wort $x$.
                    \end{itemize}
                \end{enumerate}
            
                Falls $L(M) \neq \emptyset$ existiert ein $y \in L(M)$. Dann ist $y = w_k$ für ein $k \in \N\setminus\{0\}$ und die akzeptierende Berechnung von $M$ auf $y$ hat eine endliche Länge $l$.
            
                Das Paar $(k, l)$ wird in endlich vielen Schritten erreicht und somit akzeptiert $A$ die Eingabe $x$, falls $L(M) \neq \emptyset$.
            
                Somit folgt $L(A) = L_{\text{diag}}^\complement$.
            
                $\hfill\blacksquare$
            
                \subsection{Reduktionsaufgaben}
            
                \myTitle{Aufgabe 5.22}
                Wir zeigen
                $$L \in \Lre \land L^\complement \in \Lre \iff L \in \Lr$$
            
                $\mathbf{(\implies):}$
                
                Nehmen wir $L \in \Lre \land L^\complement \in \Lre$ an.
            
                Dann existiert eine TM $M$ und $M_C$ mit $L(M) = L$ und $L(M_C) = L^\complement$.
            
                Wir konstruieren eine TM $A$, die für eine Eingabe $w$ die beiden TM's $M$ und $M_C$ parallel auf $w$ simuliert.
                
                $A$ akzeptiert $w$, falls $M$ das Wort akzeptiert und verwirft, falls $M_C$ das Wort akzeptiert.
                
                Bemerke, dass $L(M) \cap L(M_C) = \emptyset$ und $L(M) \cup L(M_C) = \Sigma^*$.
            
                Da $w \in L(M)$ oder $w \in L(M_C)$, hält $A$ immer.
            
                Da $A$ genau dann akzeptiert, falls $w \in L(M)$, folgt $L(A) = L(M) = L$.
            
                Demnach gilt $L \in \Lr$.
            
                
                $\mathbf{(\impliedby):}$
            
                Nehmen wir $L \in \Lr$ an. Per Lemma 5.4 gilt $L^\complement \leq_{\text{R}} L$ und daraus folgt auch $L^\complement \in \Lr$.
            
                Da $\Lr \subset \Lre$, folgt $L \in \Lre \land L^\complement \in \Lre$.
            
                $\hfill\blacksquare$
            
            
                \comm{Ich gebe nun zuerst ausführliche Lösungen von zwei Beispielaufgaben. Dieses Format ist für den Endterm zu lange.}


                \myTitle{Beispielaufgabe 17a HS22}

                Beweise $$L_{\text{H}} \leq_{\text{EE}} L_{\text{U}}$$
                wobei 
                $$L_{\text{H}} = \{\text{Kod}(M)\#w \mid M\text{ hält auf }w\land w \in (\Sigma_{\text{bool}})^*\}$$
                und 
                $$L_{\text{U}}=\{\text{Kod}(M)\#w \mid M \text{ akzeptiert }w \land w \in (\Sigma_{\text{bool}})^*\}$$
            
            
                \textbf{Lösung}
            
                Wir wollen $L_{\text{H}} \leq_{\text{EE}} L_{\text{U}}$ zeigen.
            
                    Wir geben die Reduktion zuerst als Zeichnung an.
            
                    \begin{figure}[htp]
                        \centering
                        \includegraphics[width = 0.7\textwidth]{Images/17a_Zeichnung.png}
                        \caption{EE-Reduktion von $L_{\text{H}}$ auf $L_{\text{U}}$}
                    \end{figure}
                    
        
                Wir definieren eine Funktion $M(x)$ für ein $x \in \{0,1,\#\}^*$, so dass
                    $$x \in L_{\text{H}} \iff M(x) \in L_{\text{U}} \qquad (1)$$
                
                    Falls $x$ nicht die richtige Form hat, ist $M(x) = \lambda$, sonst ist $M(x) = \text{Kod}(M')\#w$ wobei $M'$ gleich aufgebaut ist wie $M$, ausser dass alle Transitionen zu $q_{reject}$ zu $q_{accept}$ umgeleitet werden. Wir sehen, dass $M'$ genau dann $w$ akzeptiert, wenn $M$ auf $w$ hält. 
                    
                    Dieses $M(x)$ übergeben wir dem Algorithmus für $L_{\text{U}}$. 
            
                    
                Wir beweisen nun $x \in L_{\text{H}} \iff M(x) \in L_{\text{U}}$:
            
                    \begin{enumerate}[label=(\roman*)]
                        \item $x \in L_{\text{H}}$
            
                            Dann ist $x = \text{Kod}(M)\#w$ von der richtigen Form, und $M$ hält auf $w$. 
                            Das heisst die Simulation von $M$ auf $w$ endet entweder in $q_{reject}$ oder in $q_{accept}$. 
                            
                            Folglich wird $M'$ $w$ immer akzeptieren, da alle Transitionen zu $q_{reject}$ zu $q_{accept}$ umgeleitet wurden. 
                            
                            $x \in L_{\text{H}} \implies M(x) \in L_{\text{U}}$
            
                        \item $x \notin L_{\text{H}}$
            
                            Dann unterscheiden wir zwischen zwei Fällen:
            
                            \begin{enumerate}[label=(\alph*)]
                                \item 
                                
                                $x$ hat nicht die richtige Form, i.e. $x \neq \text{Kod}(M)\#w$.
                                Dann ist $M(x) = \lambda$ und da es keine Kodierung einer Turingmaschine $M$ gibt, so dass Kod$(M) = \lambda$, gilt   $\lambda \notin L_{\text{U}}$.
                        
            
               
                \item
            
                                $x = \text{Kod}(M)\#w$ hat die richtige Form. Dann haben wir $M(x) = $ Kod$(M')\#w$.
                                
                                Da aber $x \notin L_{\text{H}}$, hält $M$ nicht auf $w$. Da $M$ nicht auf $w$ hält, erreicht es nie $q_{reject}$ oder $q_{accept}$ in $M$ und so wird $w$ von $M'$ nicht akzeptiert. 
            
                                $\implies M(x) \notin L_{\text{U}}$
            
                               
                                
                                
                            \end{enumerate}
                            
                             So haben wir mit diesen Fällen (a) und (b) $x \notin L_{\text{H}} \implies M(x) \notin L_{\text{U}}$ bewiesen. 
                             
                             Aus indirekter Implikation folgt $M(x) \in L_{\text{U}} \implies x \in L_{\text{H}}$
                                 
                    \end{enumerate}
            
               
            
                Aus (i) und (ii) folgt 
                
                $$x \in L_{\text{H}} \iff M(x) \in L_{\text{U}} \qquad (1)$$
            
                Somit ist die Reduktion korrekt.
            
                \hspace*{0pt}\hfill$\blacksquare$
            
            
            
                \myTitle{Beispielaufgabe 18b HS22}

                Sei $$L_{\text{infinite}} = \{\text{Kod}(M) \mid \text{$M$ hält auf keiner Eingabe}\}$$
                Zeige $(L_{\text{infinite}})^C \notin \Lr$
            
            
                \textbf{Lösung}

                Wir zeigen, dass $(L_{\text{infinite}})^C \notin \Lr$ mit einer geeigneten Reduktion.
            
                Wir beweisen $L_{\text{H}} \leq_{\text{R}} (L_{\text{infinite}})^C$
            
                Um dies zu zeigen nehmen wir an, dass wir einen Algorithmus $A$ haben, der $(L_{\text{infinite}})^C$ entscheidet.
                 Wir konstruieren einen Algorithmus $B$, der mit Hilfe von $A$, die Sprache $L_{\text{H}}$ entscheidet. 
                 
                Wir betrachten folgende Abbildung:
                
                \begin{figure}[htp]
                    \centering
                    \includegraphics[width=0.75\textwidth]{Images/18b_Zeichnung.png}
                    \caption{R-Reduktion von $L_{\text{H}}$ auf $(L_{\text{infinite}})^C$}
                \end{figure}
            
            
                \begin{enumerate}[label=\Roman*.]
                    \item Für eine Eingabe $x \in \{0,1,\#\}^*$ berechnet das Teilprogramm $C$, ob $x$ die richtige Form hat(i.e. ob $x = $ Kod$(M)\#w$ für eine TM $M$).
                    \item Falls nicht, verwirft $B$ die Eingabe $x$.
                    
                    \item Ansonsten, konstruiert $C$ eine Turingmaschine $M'$, die Eingaben ignoriert und immer $M$ auf $w$ simuliert. Wir sehen, dass $M'$ genau dann hält, wenn $M$ auf $w$ hält. 
                    \item Folglich hält $M'$ entweder für jede Eingabe ($M$ hält auf $w$) oder für keine ($M$ hält nicht auf $w$).
                     
                    \item Da $A$ genau dann akzeptiert, wenn die Eingabe keine gültige Kodierung ist(ausgeschlossen, da $C$ das herausfiltert) oder wenn die Eingabe $M(x)= \text{Kod}(M')$ und $M'$ für mindestens eine Eingabe hält, akzeptiert $A$ $M(x)$ genau dann, wenn $x = \text{Kod}(M)\#w$ die richtige Form hat und $M$ auf $w$ hält.
                \end{enumerate}
                
                
                Folglich gilt $$x \in L_{\text{H}} \iff M(x) \in (L_{\text{infinite}})^C$$
                $\implies L_{\text{H}} \leq_R (L_{\text{infinite}})^C$ 
            
                Also folgt die Aussage
                $$(L_{\text{infinite}})^C \in \L_R \implies L_{\text{H}} \in \L_R$$
            
                Da wir $L_{\text{H}} \notin \L_R$ (\textbf{Satz 5.8}), folgt per indirekter Implikation:
                $$(L_{\text{infinite}})^C \notin \L_R$$
            
                \hspace*{0pt}\hfill$\blacksquare$

                \comm{Hier ist nun eine minimale Lösung für eine Reduktionsaufgabe.}

                    \myTitle{Minimale Aufgabe 1}
                    Zeige 
                    $$L_{\text{diag}} \leq_{\text{EE}} L_{\text{H}}^\complement$$
                    
                    Zur Erinnerung:
                    $$L_{\text{diag}} = \{w_i \in (\Sigma_{\text{bool}})^* \mid M_i \text{ akzeptiert } w_i \text{ nicht}\}$$
                    \begin{align*}
                        L_{\text{H}}^\complement &= \{\text{Kod}(M)\#w \in \{0,1 , \#\}^* \mid M \text{ hält nicht auf }w\}\\
                                                & \cup \{x \in \{0, 1, \#\}^* \mid x \text{ nicht von der Form Kod}(M)\#w\}
                    \end{align*}
                
                
                
                    \myTitle{Minimale Lösung 1}
                    Wir beschreiben einen Algorithmus $A$, so dass 
                    $$x \in L_{\text{diag}} \iff A(x) \in L_{\text{H}}^\complement$$
                    \textbf{Eingabe:} $x \in (\Sigma_{\text{bool}})^*$
                    
                    \begin{enumerate}[label=\arabic*.]
                        \item Findet $i$ so dass $x = w_i$
                        \item Generiert Kod$(M_i)$
                        \item Generiert Kod$(\overline{M}_i)$ mit folgenden Modifikationen zu Kod$(M_i)$
                        
                        \begin{itemize}[label=-]
                            \item Transitionen nach $q_{\text{reject}}$ werden in eine Endlosschleife umgeleitet.
                        \end{itemize}
                        \item Gibt Kod$(\overline{M}_i)\#w_i$ aus.
                    \end{enumerate}
            
                    \textbf{Case Distinction}
                    \begin{enumerate}[label=\Roman*.]
                        \item $\mathbf{x \in L_{\text{diag}}}$
                        \begin{center}
                            $\implies$ $M_i$ akzeptiert $x = w_i$ nicht\\
                            $\implies$ $\overline{M}_i$ hält nicht auf $w_i$\\
                            $\implies$  $A(x) = \text{Kod}(\overline{M}_i)\#w_i \in L_{\text{H}}^\complement$
                        \end{center}
                        
                        \item $\mathbf{x \notin L_{\text{diag}}}$
                        \begin{center}
                            $\implies$ $M_i$ akzeptiert $x = w_i$\\
                            $\implies$ $\overline{M}_i$ hält auf $w_i$\\
                            $\implies$  $A(x) = \text{Kod}(\overline{M}_i)\#w_i \notin L_{\text{H}}^\complement$
                        \end{center}
                    \end{enumerate}
                    $\hfill\blacksquare$

            \comm{Hier gibt es noch eine Aufgabe, bei der die Kernidee ein wenig schwieriger ist. Wir generieren die Kodierung einer TM, die nicht immer das gleiche macht,
                sondern die Länge ihrer eigenen Eingabe verwendet.}

            \myTitle{Aufgabe 2}
            Sei $L_{\text{all}} = \{\text{Kod}(M) \mid M \text{ akzeptiert jede Eingabe}\}$.

	Zeigen Sie $L_{\text{H}}^\complement \leq_{\text{EE}} L_{\text{all}}$.
	
	%\pause
	\textbf{Kernidee}

	Für eine Eingabe $x = \text{Kod}(M)\#w$, generieren wir Kod$(A)$ einer TM $A$, die folgendes folgendes macht:
	
	$\mathbf{A}(y)$:
	\begin{enumerate}[label=\arabic*.]
		
		\item Berechnet $|y|$ Schritte von $M$ auf $w$.
		%\pause
		\item Falls danach die Berechnung nach $|y|$ noch nicht terminiert hat, akzeptiert $A$ die Eingabe $y$.
		
		\item Sonst verwirft $A$ die Eingabe.
	\end{enumerate}
	%\pause
	\begin{center}
		$A$ akzeptiert jede Eingabe $\iff$ $M$ läuft unendlich auf $w$\\
		\(\text{Kod}(A) \in L_{\text{all}} \iff \text{Kod}\#w \in L_{H}^\complement\)
	\end{center}

    \myTitle{Beispielaufgabe: Satz von Rice}
    Wir definieren $$L_{\text{all}} = \{\text{Kod}(M) \mid L(M) = (\Sigma_{\text{bool}})^*\}.$$

    Zeige $$L_{\text{all}} \notin \Lr.$$
       
    
     
    \subsection{Satz von Rice - Beweis}
                
                
    \myTitle{Spezialfall des Halteproblems}
    
    Wir definieren $L_{\text{H}, \lambda} = \{\text{Kod}(M) \mid M \text{ hält auf }\lambda\}$.
    \begin{mainbox}{Lemma 5.8}
        $$L_{\text{H}, \lambda} \notin \Lr$$
    \end{mainbox}
    \textbf{Beweis: }

    Wir zeigen $L_\text{H} \leq_\text{EE} L_{\text{H}, \lambda}$. Wir beschreiben einen Algorithmus $B$, so dass $x \in L_\text{H} \iff B(x) \in L_{\text{H}, \lambda}$.

    Für jede Eingabe arbeitet $B$ wie folgt:
    \begin{itemize}
        \item Falls $x$ von der falschen Form, dann $B(x) = M_{inf}$, wobei $M_{inf}$ unabhängig von der Eingabe immer unendlich läuft.
        \item Sonst $x = \text{Kod}(M)\#w$: Dann $B(x) = M'$, wobei $M'$ die Eingabe ignoriert und immer $M$ auf $w$ simuliert.
    \end{itemize}

    Wir sehen, dass $M'$ genau dann auf $\lambda$ hält, wenn $x \in L_{\text{H}}$.

    Daraus folgt $x \in L_\text{H} \iff B(x) \in L_{\text{H}, \lambda}$.

    \hspace*{0pt}\hfill$\blacksquare$

    \begin{mainbox}{Satz 5.9 (Rice)}
        Jedes semantisch nichttriviale Entscheidungsproblem über Turingmaschinen ist unentscheidbar.
    \end{mainbox}
    


        \subsubsection{Prerequisites}
        \begin{mainbox}{Semantisch nichttriviales Entscheidungsproblem über TMs}
            Das Entscheidungsproblem $(\Sigma, L)$, bzw. die Sprache $L$ muss folgendes erfüllen.
            \begin{enumerate}[label=\Roman*.]
                
                \item $L \subseteq \textbf{KodTM}$
                
                \item $\exists M_1$ so dass $\text{Kod}(M_1) \in L (\text{i.e. } L \neq \emptyset)$
                \item $\exists M_2$ so dass $\text{Kod}(M_2) \notin L (\text{i.e. } L \neq \textbf{KodTM})$
                
                \item Für zwei TM $A$ und $B$ mit $L(A) = L(B)$ gilt
                $$\text{Kod}(A) \in L \iff \text{Kod}(B) \in L$$
            \end{enumerate}
        \end{mainbox}
        \begin{itemize}
            \item 'über Turingmaschinen' = $L \subseteq \textbf{KodTM}$.
            \item 'nichttrivial' = $\exists M_1: \text{Kod}(M_1) \in L$ und $\exists M_2: \text{Kod}(M_2) \notin L$
            \item 'semantisch' = Für $A, B$ mit $L(A) = L(B)$ gilt $\text{Kod}(A) \in L \iff \text{Kod}(B) \in L$.
        \end{itemize}
        $\textbf{KodTM}\subseteq (\Sigma_{\text{bool}})^*$ ist die Menge aller Kodierungen von Turingmaschinen.
    
    
        \subsubsection{Idee}
        Wir zeigen für jedes \textbf{semantisch nichtriviale Entscheidungsproblem} $(\Sigma, L)$
        
        $$L \in \Lr \implies L_{H, \lambda} \in \Lr$$
        
        Aus dem folgt dann per Kontraposition 
    
        $$L_{H, \lambda} \notin \Lr \implies L \notin \Lr$$
        
        Mit der Aussage $L_{H, \lambda} \notin \Lr$ von \textbf{Lemma 5.8}, können wir dann 
    
        $$L \notin \Lr$$
        wie gewünscht folgern.
    
    
        Wir müssen dann nur noch die Implikation 
        $$L \in \Lr \implies L_{H, \lambda} \in \Lr$$
        beweisen.
        
    
        \textbf{Kernidee}
        \begin{center}
            Wir zeigen die \textbf{Existenz} einer Reduktion, aus der die Implikation folgt. 
        \end{center}
        
        Konkret machen wir eine \textbf{Case Distinction} und zeigen jeweils 
        \begin{itemize}[label=-]
            \item Die \textbf{Existenz} einer EE-Reduktion von $L_{H, \lambda}$ auf  $L$
            
            Daraus folgt $L_{H, \lambda} \leq_{\text{EE}} L$.
            \item oder die \textbf{Existenz} einer EE-Reduktion $L_{H, \lambda} $ auf $ L^\complement$
            
            Daraus folgt $L_{H, \lambda} \leq_{\text{EE}} L^\complement$.
        \end{itemize}
        Zur Erinnerung:
    
        \begin{mainbox}{Lemma 5.3}
            Seien $L_1 \subseteq \Sigma_{1}^*$ und $L_2 \subseteq \Sigma_{2}^*$ zwei Sprachen.
            $$L_1 \leq_{\text{EE}} L_2 \implies L_1 \leq_{\text{R}} L_2$$
        \end{mainbox}
    
        Weshalb reicht es $L_{H, \lambda} \leq_{\text{EE}} L^\complement$ zu zeigen?
    
        
        \begin{mainbox}{Lemma 5.4}
            Sei $\Sigma$ ein Alphabet. Für jede Sprache $L \subseteq \Sigma^*$ gilt:
         $$L \leq_\text{R} L^\complement \text{ und } L^\complement \leq_\text{R} L$$
        \end{mainbox}
    
        In beiden Cases folgt mit \textbf{Lemma 5.3} und \textbf{Lemma 5.4}, die gewünschte Aussage $L_{H, \lambda} \leq_{\text{R}} L$.
    
        Explizit gilt nun
    
        \begin{enumerate}[label=\arabic*.]
            \item $$L_{H, \lambda} \leq_{\text{EE}} L^\complement \ \xRightarrow{\textbf{Lemma 5.3}} \ L_{H,\lambda} \leq_{\text{R}} L^\complement \ \xRightarrow{\textbf{Lemma 5.4}} \ L_{H,\lambda} \leq_{\text{R}} L$$
            \item $$L_{H, \lambda} \leq_{\text{EE}} L \ \xRightarrow{\textbf{Lemma 5.3}} \ L_{H,\lambda} \leq_{\text{R}} L$$
        \end{enumerate}
    
        
        Aus $L_{H,\lambda} \leq_{\text{R}} L$ folgt (in beiden Cases) die gewünschte Implikation  
        $$L \in \Lr \implies L_{H, \lambda} \in \Lr$$
    
    
    
        \subsubsection{Beweis}
        Sei $M_{\emptyset}$ eine TM s.d. $L(M_{\emptyset}) = \emptyset$. 
    
        \textbf{Case Distinction}
        \begin{enumerate}[label=\Roman*.]
            \item $\mathbf{\textbf{Kod}(M_{\emptyset}) \in L}$
            
            Wir zeigen $L_{H, \lambda} \leq_{\text{EE}} L^\complement$.
            \item $\mathbf{\textbf{Kod}(M_{\emptyset}) \notin L}$
            
            Wir zeigen $L_{H, \lambda} \leq_{\text{EE}} L$.
        \end{enumerate}
    
    
    
        \myTitle{Case I.  $\mathbf{\textbf{Kod}(M_{\emptyset}) \in L}$}
        
        Es \textbf{existiert} eine TM $\overline{M}$, so dass Kod$(\overline{M}) \notin L$. (Nichttrivialität)
    
        
        Wir beschreiben eine TM $S$, so dass für eine Eingabe $x \in (\Sigma_{\text{bool}})^*$
        $$x \in L_{H, \lambda} \iff S(x) \in L^\complement$$
    
        Daraus folgt dann die gewünschte EE-Reduktion.
    
        
        Wir verwenden dabei $M_{\emptyset}$ und $\overline{M}$, da $\text{Kod}(M_{\emptyset}) \notin L^\complement$ und Kod$(\overline{M}) \in L^\complement$.
    
    
    
        \myTitle{Case I.  $\mathbf{\textbf{Kod}(M_{\emptyset}) \in L}$ - Beschreibung von $S$}

        \textbf{Eingabe} $x \in (\Sigma_{\text{bool}})^*$
    
        \begin{enumerate}[label=\arabic*.]
            
            \item $S$ überprüft ob $x = \text{Kod}(M)$ für eine TM $M$.
            
            Falls dies \textbf{nicht} der Fall ist, gilt $S(x) = \text{Kod}(M_{\emptyset})$
            
            \item Sonst $x = \text{Kod}(M)$. Dann $S(x) = \text{Kod}(A)$, wobei $A$ wie folgt kodiert ist.
            \begin{enumerate}[label=\roman*.]
                \item Gleiches Eingabealphabet wie $\overline{M}$, i.e. $\Sigma_A = \Sigma_{\overline{M}}$.
                
                \item Für eine beliebige Eingabe $y \in (\Sigma_{\overline{M}})^*$, 
                simuliert $A$ zuerst $M$ auf $\lambda$ \textbf{ohne die Eingabe $y$ zu überschreiben}.
                
                \item Danach simuliert $A$ die TM $\overline{M}$ auf die gegebene Eingabe $y$.
                \item Akzeptiert $y$ genau dann, wenn $\overline{M}$ $y$ akzeptiert.
            \end{enumerate}
        \end{enumerate}
    
    
    
        \myTitle{Korrektheit}

        Wir zeigen 
        $$x \in L_{H, \lambda} \iff S(x) \in L^\complement$$
        $\mathbf{(\implies):}$
        
        Wir nehmen \textbf{$x \in L_{H, \lambda}$} an und zeigen $S(x) \in L^\complement$.
        
        
        Da $M$ auf $\lambda$ hält, wird $A$ immer $\overline{M}$ auf der Eingabe $y$ simulieren und wir haben $L(A) = L(\overline{M})$.
    
        
        Da $L$ (und somit auch $L^\complement$) ein \textbf{semantisches} Entscheidungsproblem ist, gilt 
        $$\text{Kod}(\overline{M}) \in L^\complement \implies \text{Kod}(A) \in L^\complement$$
        Da die LHS der Implikation gegeben ist, folgt $S(x) = \text{Kod}(A) \in L^\complement$
            
        $\mathbf{(\impliedby):}$
    
        Wir nehmen $x \notin L_{H, \lambda}$ an und zeigen $S(x) \notin L^\complement$. 
    
        Aus Kontraposition folgt dann die gewünschte Rückimplikation.
    
        
        Da $M$ nicht auf $\lambda$ hält, wird $A$ bei jeder Eingabe nicht halten.
    
        
        Somit folgt $L(A) = L(M_{\emptyset})$ und da $\text{Kod}(M_{\emptyset}) \notin L^\complement$ per semantische Eigenschaft von $L$
        $$S(x) = \text{Kod}(A) \notin L^\complement$$
    
    
    
        \myTitle{Case II.}

        Zweite Case funktioniert genau gleich.
    
        Wir haben Kod$(M_{\emptyset}) \notin L$. 
        
        Per Nichttrivialität existiert eine TM $\overline{M}$ mit Kod$(\overline{M}) \in L$.
        
        ...
        
        $\hfill\blacksquare$
    
        \subsection{EE Reduktion angewendet für $\Lre$}

            EE-Reduktion impliziert RE-Reduktion (\textbf{nicht in der Vorlesung})
            \begin{mainbox}{Lemma zu RE-Reduktion}
                $$L_1 \leq_{\text{EE}} L_2 \implies (L_2 \in \Lre \implies L_1 \in \Lre)$$
            \end{mainbox}
            \textbf{Beweis}
        
            Sei $L_1 \leq_{\text{EE}} L_2$ und $L_2 \in \Lre$.
        
            Wir zeigen nun $L_1 \in \Lre$.
        
            Per Definition von $L_1 \leq_{\text{EE}} L_2$ existiert ein Algorithmus $F$, 
            der die Funktion $f:\Sigma_1^* \to \Sigma_2^*$ berechnet, so dass
        
            $$\forall x \in \Sigma_1^*. x \in L_1 \iff f(x) \in L_2$$
        
            Da $L_2 \in \Lre$ existiert eine TM $M_2$ (die nicht unbedingt immer terminiert) mit $L(M_2) = L_2$.
        
            Wir beschreiben mit $F$ und $M_2$ nun eine TM $M_1$ mit $L(M_1) = L_1$.
            
            \textbf{Eingabe:} $x \in \Sigma_1^*$
            \begin{enumerate}[label=\arabic*.]
                \item $F$ berechnet auf $x$ und übergibt seine Ausgabe $f(x)$ zur TM $M_2$
                \item $M_2$ berechnet auf $f(x)$ und die Ausgabe wird übernommen.
            \end{enumerate}
        
            \begin{figure}[htp]
                \includegraphics[width = 0.75\textwidth]{Images/RE-Reduzierbarkeit.png}
                \caption{TM $M_1$, Zsf. Fabian Frei}
            \end{figure}
        
            \textbf{Korrektheit} ($L_1 = L(M_1)$)
            
            \textbf{Case Distinction}
            \begin{enumerate}[label=\Roman*.]
                \item $\mathbf{x \in L_1}$
                
                $\implies f(x) \in L_2$  (Algorithmus $F$ terminiert immer)
        
                $L(M_2) = L_2 \implies f(x) \in L(M_2)$ 
                
                da die Ausgabe von $M_2$ übernommen wird
                
                $\implies x \in L(M_1)$
                \item $\mathbf{x \notin L_1}$
                
                $\implies f(x) \notin L_2$
        
                $\implies f(x) \notin L(M_2)$
        
                $\implies x \notin L(M_1)$
            \end{enumerate}
            $\hfill\blacksquare$
        
            \subsection{Verhältnis zwischen RE 'Reduktion' und R-Reduktion}
            
            $$L_1 \leq_{\text{R}} L_2 \centernot{\implies} (L_2 \in \Lre \implies L_1 \in \Lre)$$
            Wir beweisen diese Aussage per Gegenbeispiel. 
        
            Sei $L_1 = L_{\text{diag}}$ und $L_2 = L_{\text{diag}}^\complement$.
        
            Wir haben 
            \begin{itemize}[label=$\blacktriangleright$]
                \item $L_1 = L_{\text{diag}} \notin \Lre$ \hfill (Satz 5.5)
                \item $L_2 = L_{\text{diag}}^\complement \in \Lre\setminus\Lr$ \hfill(Korollar 5.2, Lemma 5.5)
            \end{itemize}
            
            Per \textbf{Lemma 5.4} gilt $L_{\text{diag}} \leq_{\text{R}} L_{\text{diag}}^\complement$.
        
            Die rechte Implikation gilt jedoch nicht.
            
            $\hfill\blacksquare$
        
            $$L_1 \leq_{\text{R}} L_2 \centernot{\impliedby} (L_2 \in \Lre \implies L_1 \in \Lre)$$
            Sei $L_1 = L_{\text{U}}$ und $L_2 = \{0^i \mid i \in \N\}$.
        
            Wir haben 
            \begin{itemize}[label=$\blacktriangleright$]
                \item $L_1 = L_{U} \in \Lre \setminus \Lr$ \hfill (Satz 5.6 und 5.7)
                \item $L_2 = \{0^i \mid i \in \N\} \in \Lr$ \hfill(da $\L_{\text{EA}} \subset \Lr$)
            \end{itemize}
            
            Da $L_1 \in \Lre$, gilt die Implikation auf der rechten Seite für dieses $L_1$ und $L_2$.
        
            
            Da per Definition
            $$L_1 \leq_{\text{R}} L_2 \iff (L_2 \in \Lr \implies L_1 \in \Lr)$$
            folgt aus $L_1 \notin \Lr$ und $L_2 \in \Lr$, dass diese Instanzierung von $L_1$ und $L_2$ ein Gegenbeispiel ist.
            
            $\hfill\blacksquare$
        
            \myTitle{Relation zu EE-Reduktion}

            Wir haben aber gezeigt, dass 
            $$L_1 \leq_{\text{EE}} L_2 \implies L_1 \leq_{\text{R}} L_2$$
            und 
            
            $$L_1 \leq_{\text{EE}} L_2 \implies (L_2 \in \Lre \implies L_1 \in \Lre)$$
        
            Die Rückrichtung gilt jeweils nicht.